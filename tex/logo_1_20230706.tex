\documentclass[tikz]{standalone}
\usetikzlibrary{fadings,patterns,fit}

% use stamp.png to make a fading (a mask)
\begin{tikzfadingfrompicture}[name=stamp]
  \node [fill=transparent!0,inner sep=0]
  {\includegraphics[width=50pt,height=50pt]{trans_5.png}};
  \begin{scope}[yshift=-90pt,transform canvas={scale=.2}]
    \node[color=gray!50!black,font=\ttfamily] {\today};
  \end{scope}
\end{tikzfadingfrompicture}

\colorlet{mycolor}{blue}
\pgfdeclarefunctionalshading{stampfunctional}
{\pgfpointorigin}{\pgfpoint{4cm}{4cm}}{\pgfshadecolortorgb{mycolor}{\myrgb}}{
  20 mul
  sin 1 add 0.5 mul
  exch
  20 mul
  cos 1 add 0.5 mul
  add 0.5 mul
  %1 exch sub
  dup \myrgbred
  1 exch sub mul 1 exch sub exch
  dup \myrgbgreen
  1 exch sub mul 1 exch sub exch
  \myrgbblue
  1 exch sub mul 1 exch sub
}

\begin{document}
\begin{tikzpicture}[inner sep=0]
  \colorlet{mycolor}{blue!70!black}
  % fill a region stampshading
  \node[
  shading=stampfunctional,
  shading angle=-45,
  minimum width=4cm,minimum height=4cm]{};
  % fade the same region using stamp fading as mask
  \node[fill=white,minimum width=4cm,minimum height=4cm,path fading=stamp]{};

  \colorlet{mycolor}{lime!70!black}
  % fill a region stampshading
  \node[
  shading=stampfunctional,
  shading angle=-45,
  minimum width=4cm,minimum height=4cm] at (0,4){};
  % fade the same region using stamp fading as mask
  \node[fill=white,minimum width=4cm,minimum height=4cm,path fading=stamp] at (0,4){};
\end{tikzpicture}
\end{document}
